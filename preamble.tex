% This software has been dedicated to the public domain under CC0:
% https://creativecommons.org/public-domain/cc0/

\documentclass[b5paper, 10pt, twoside]{book}

\frenchspacing

% Programming tools
\usepackage{ifthen}     % conditional commands
\usepackage{etoolbox}   % tools for intefacing with internals

\newcommand{\outformat}{electronic}

% Sizes of inner and outer margins
\ifthenelse{\equal{\outformat}{print}}{
    \newcommand{\innersize}{55.4pt}
    \newcommand{\outersize}{110.8pt}
}{
    \newcommand{\innersize}{83.1pt}
    \newcommand{\outersize}{83.1pt}
}

% Sizes of top and bottom margins
\newcommand{\topsize}{55.4pt}
\newcommand{\bottomsize}{115.2pt}

% Page layout
\usepackage[
    inner = \innersize,
    outer = \outersize,
    top = \topsize,
    bottom = \bottomsize
]{geometry} % modify page layout

% Debugging
%\usepackage{showframe}     % show the page layout grid
%\usepackage{kantlipsum}    % English placeholder text

% Mathematics
\usepackage{amsmath}    % various math macros
\usepackage{mathtools}  % more math
%\usepackage{amsthm}    % theorem/definition/etc. enivornments
\usepackage{amssymb}    % more math symbols
\usepackage{slashed}    % slashed symbols

% Miscellaneous symbols
\usepackage{gensymb}    % adds degree symbol (and other stuff)
\usepackage{ccicons}    % creative commons icons
\usepackage{textgreek}  % macros for greek letters in text

% General styling
\usepackage{emptypage}  % suppress page numbers on empty pages
\usepackage{fancyhdr}   % fancy header designs
\usepackage{titlesec}   % modify appearance of section titles
\usepackage{titletoc}   % customize table of contents

\usepackage{graphicx}   % better graphics

% Text formatting
%\usepackage{setspace}                   % options for line spacing
\usepackage[xspace]{ellipsis}           % fixes spacing around ellipses
\usepackage[letterspace=70]{microtype}  % microtypographic features

% Localization
\usepackage[english]{babel}

% Float customization
\usepackage[
    format = plain, labelfont = bf
]{caption}                  % customize caption formatting
\usepackage{multirow}       % rows spanning multiple columns for tables
\usepackage{tabularray}     % tables using latex3
\usepackage{threeparttable} % tables with notes
\usepackage{tabularx}       % adjust table column widths
\usepackage{booktabs}       % more professional tables

% Vector graphics
\usepackage{pgf}            % backend for tikz and others
\usepackage{tikz}           % generate graphics programmatically
\usepackage{pgfplots}       % plots using pgf
\usepackage{tikz-3dplot}    % easier 3d plots in tikz
%\usepackage{tikz-feynman}  % easy fenyman diagrams

% Font/text backend stuff
%\usepackage[utf8]{inputenc}                % input encodings (no longer needed)
\usepackage[T1]{fontenc}                    % font encodings
\usepackage{fontspec}                       % interface for opentype fonts
\usepackage[bold-style=ISO]{unicode-math}   % fontspec equivalent for math fonts

% Colors
\usepackage{xcolor} % facilitates color in documents

% Miscellaneous packages
%\usepackage{marginnote}        % allows margin notes
\usepackage[inline]{enumitem}   % more options for enumerate environment
\usepackage{hyperref}           % in-document hyperlink references
\usepackage{zref-totpages}      % macro for printing total number of pages

% Custom packages
\usepackage{scpcolors}      %  collection of custom colors
\usepackage{journalnames}   %  journal name macros

\definecolor{grey}{gray}{0.3}

% It's hard to make hyperlinks look tasteful.
\newcommand{\typographersred}{scp-red-dark-3}
\newcommand{\linkblue}{scp-blue-dark-3}

\ifthenelse{\equal{\outformat}{electronic}}
{
    \newcommand{\linkcolor}{\typographersred}
    \newcommand{\citecolor}{\typographersred}
    \newcommand{\urlcolor}{\linkblue}
}{
    \newcommand{\linkcolor}{black}
    \newcommand{\citecolor}{black}
    \newcommand{\urlcolor}{black}
}

\hypersetup{
    linktocpage,
    colorlinks = true,
    linkcolor = \linkcolor,
    citecolor = \citecolor,
    urlcolor = \urlcolor}


% Biblatex data model config file. If you find yourself needing new custom 
% fields in your `.bib` file, put them here.
\begin{filecontents*}{biblatex-dm.cfg}
\DeclareDatamodelFields[type=field, datatype=literal]{collaboration}
\DeclareDatamodelEntryfields{collaboration}
\end{filecontents*}

% biblatex setup
%
% `collabauthoryear` is a custom cite/bibstyle. It's basically `authoryear`, but
% but in citations it uses the `collaboration` field instead of author, if
% available, and in the bibliougraphy it prints the `collaboration` field in
% parentheses, if available. The definitions are found in the respective `.cbx`
% and `.bbx` files.
\usepackage[
    backend = biber,
    language = english,
    natbib = true,
    citestyle = collabauthoryear-comp,
    bibstyle = collabauthoryear,
    sorting = nyt,
    giveninits = true,
    minbibnames = 3,
    mincitenames = 1,
    maxbibnames = 3,
    maxcitenames = 3,
    hyperref = true,
    arxiv = abs,
    uniquename = full
]{biblatex}

% These are some modifications to bibliography entry styling:
%   - Remove "in:" before journal name.
%   - Use the 3-em dash for repeat authors.
%   - Set spacing between name initials.
%   - Add interword spacing to acronyms.
\renewbibmacro{in:}{}
\renewcommand*\bibnamedash{\rule[0.48ex]{3em}{0.14ex}\space}
\renewcommand*\bibnamedelimi{\hspace{0.125em}}
\renewcommand*{\mkbibacro}[1]{\textls[70]{#1}}

% Fields to exclude from bibliography
\AtEveryBibitem{\clearfield{url}}
\AtEveryBibitem{\clearfield{issn}}

% Allow line breaks after lower case letters in URLs and DOIs in bibliography.
% The high penalty ensures that this is only allowed in the rare event the
% URL/DOI overflows into the margin.
\setcounter{biburllcpenalty}{9000}

\DeclareFieldFormat{authortype}{\mkbibparens{#1}}
\DeclareFieldFormat{pubtitle}{\textsb{#1}}

% Citation command for citing included publications
\DeclareCiteCommand{\pubcite}[\AtNextCitekey{\defcounter{maxnames}{99}}]
    {\usebibmacro{prenote}}
    {\textbf{\printfield[pubtitle]{title}}\\
        \printnames[][1-]{author}\\
        \printfield{year}\addcomma\addspace
        \printfield{journaltitle}
        \printfield{volume}\addperiod
        \printfield{number}\addcomma\addspace
        \iffieldundef{eid}
            {}
            {\printfield{eid}\addcomma}
        \iffieldundef{pages}
            {}
            {\printfield{pages}\addcomma}
        \printfield[eprint:arxiv]{eprint}\addperiod}
    {\multicitedelim}
    {\usebibmacro{postnote}}

% Special handling of NASA/ADS eprints
\DeclareFieldFormat{eprint:ads}{%
    {\mkbibacro{ADS}\addcolon}\space%
    \href{https://ui.adsabs.harvard.edu/abs/#1/abstract}{%
        \nolinkurl{#1}%
    }%
}%

% Special handling of IERS technical note eprints
\DeclareFieldFormat{eprint:ierstn}{%
    {\mkbibacro{IERS}\addcolon}\space%
    \href{https://www.iers.org/IERS/EN/Publications/TechnicalNotes/tn#1.html}{%
        \iffieldundef{eprinttype}{}{\texttt{tn}}\nolinkurl{#1}%
    }%
}

% Bibliography file
\addbibresource{thesis.bib}

\usetikzlibrary{decorations.markings}   % decorations
\usetikzlibrary{arrows.meta}            % more arrow styles
\usetikzlibrary{calc}			        % complex coordinate calculations
\usetikzlibrary{positioning}            % don't remember what this does
\usetikzlibrary{3d}						% 3D coordinates
\usetikzlibrary{graphs}					% graph drawing
\usetikzlibrary{shapes.geometric}       % various geometric shapes

\pgfplotsset{compat=newest}         % needed for newest plot features

\usepgfplotslibrary{fillbetween}    % fill between lines
\usepgfplotslibrary{groupplots}     % grids of plots
\usepgfplotslibrary{colormaps}      % additional colormaps
\usepgfplotslibrary{extracolormaps} % even more colormaps (custom library)

% The default tick color in PGFPlots is grey for some reason. Make it black.
\pgfplotsset{every tick/.append style={color=black}}

% Font settings
%
% Note: the fonts used need to be installed on the system. All these fonts 
% should be included in a modern TeXLive distribution. If not, install the 
% fonts f the fonts and run `luaotfload-tool -u` to update the font database.
\setmainfont{STIXTwoText}[BoldFont=Stix Two Text Semibold]
\setsansfont{OpenSans}
\setmonofont{NotoMono}[Scale=0.91]
\setmathfont{StixTwoMath}[StylisticSet=1]

% Semibold fonts
\newfontface\sbstyle{StixTwoText-Semibold}
\newfontface\sbsfstyle{OpenSans-Semibold}

% Some special styles
\newcommand{\titlestyle}{\sbsfstyle}
\newcommand{\annotatestyle}{\sbsfstyle}
\newcommand{\headstyle}{\sbsfstyle}
\newcommand{\subheadstyle}{\sbsfstyle}
\newcommand{\subsubheadstyle}{\sbsfstyle}

% Additional commands for typesetting semibold (sb) and uppercase (uc) text in
% serif and sans serif (sf) style
\newcommand{\textsb}[1]{{\sbstyle #1}}
\newcommand{\textsbsf}[1]{{\sbsfstyle #1}}
\newcommand{\textuc}[1]{\textls[60]{#1}}
\newcommand{\textucsf}[1]{\textls[60]{\textsf{#1}}}

\SetTblrStyle{remark-tag}{font=\itshape}
\SetTblrStyle{remark-sep}{font=\itshape}
\SetTblrStyle{caption-tag}{font=\bfseries}
\SetTblrStyle{caption}{halign=l}

% Redefine vector command to use bold italics
\renewcommand{\vec}[1]{\symbfit{#1}}

% Special typesetting for annotating plots
\newcommand{\plotlineannotation}[1]{{\scriptsize\annotatestyle #1}}

% Definitions for part, chapter, section, subsection styles
\titleformat{\part}[display]
    {\filcenter\huge\bfseries}{Part \thepart}{1pc}{\vspace{1pc}}
\titleformat{\chapter}[hang]
    {\Large\headstyle\filright}{\thechapter}{0.5em}{}[\vspace{4pt}]
\titleformat{\section}[hang]
    {\large\subheadstyle\filright}{\thesection}{0.5em}{}
\titleformat{\subsection}[hang]
    {\subsubheadstyle\filright}{\thesubsection}{0.5em}{}

% Spacing around headings
%\titlespacing*{\part}{0em}{2pc}{0em}
\titlespacing*{\chapter}{0em}{2pc}{*2.5}
\titlespacing*{\section}{0em}{*2.0}{*2.0}
\titlespacing*{\subsection}{0em}{*2.0}{*2.0}

% Table of contents appearance
\contentsmargin{2.55em}
\dottedcontents{section}[3.8em]{}{2.3em}{1pc}
\dottedcontents{subsection}[6.1em]{}{3.2em}{1pc}

% Redefine plain page style to put page numbers on the outer edge in `twoside`
% mode.
\ifthenelse{\equal{\outformat}{print}}{
    \fancypagestyle{plain}{%
        \fancyhf{}%
        \fancyfoot[LE,RO]{\thepage}%
        \renewcommand{\headrulewidth}{0pt}%
    }
    \pagestyle{plain}
}{
    \pagestyle{plain}
}

% Vector differential operators typeset like vectors
\newcommand{\Nabla}{\vec{\nabla}}
\newcommand{\Div}{\Nabla\cdot}
\newcommand{\Curl}{\Nabla\times}

% Derivatives
\newcommand{\tder}[2]{d#1/d#2}
\newcommand{\der}[2]{\frac{d#1}{d#2}}
\newcommand{\dder}[2]{\frac{d^2#1}{d#2^2}}
\newcommand{\ddder}[3]{\frac{d^2#1}{d#2d#3}}
\newcommand{\dern}[3]{\frac{d^{#3}#1}{d#2^{#3}}}
\newcommand{\inder}[2]{\frac{\mathfrak{d}#1}{\mathfrak{d}#2}}

% Partial derivatives
\newcommand{\pder}[2]{\frac{\partial#1}{\partial#2}}
\newcommand{\pdder}[2]{\frac{\partial^2#1}{\partial#2^2}}
\newcommand{\ppder}[3]{\frac{\partial^2#1}{\partial#2\partial#3}}
\newcommand{\pdern}[3]{\frac{\partial^{#3}#1}{\partial#2^{#3}}}
\newcommand{\tpder}[2]{\partial#1/\partial#2}

% Miscellaneous commands
\newcommand{\unitv}[1]{\symbfit{\hat{#1}}}
\newcommand{\difd}{\,d}
\newcommand{\mean}[1]{\left\langle#1\right\rangle}
\newcommand{\tmean}[1]{\langle#1\rangle}
\newcommand{\vblank}{\vspace{1pc}}
\newcommand{\imp}[1]{\textsf{#1}}
\newcommand{\lssc}[1]{\textls[70]{\textsc{#1}}}
\newcommand{\transp}{\mathsf{T}}
\newcommand{\mand}{\quad\text{and}\quad}

% Bra-ket macros
\newcommand{\bra}[1]{\left\langle#1\right|}
\newcommand{\ket}[1]{\left|#1\right\rangle}
\newcommand{\braket}[2]{\left\langle#1\middle|#2\right\rangle}
\newcommand{\brakett}[3]{\left\langle#1\middle|#2\middle|#3\right\rangle}
\newcommand{\redbrakett}[3]{\left\langle#1\middle\|#2\middle\|#3\right\rangle}

% Command for ignoring large blocks of text. Saves your pinky for RSI.
\newcommand{\nothing}[1]{}

% TODO commands
\newcommand{\needcite}{\textcolor{\typographersred}{[citation needed]}}
\newcommand{\cited}[1]{\textcolor{\typographersred}{[#1]}}
\newcommand{\todo}[1][]{%
   \ifthenelse{\equal{#1}{}}
        {\textcolor{\typographersred}{[TODO]}}
        {\textcolor{\typographersred}{[TODO: #1]}}
}

% Some negative spacing commands
\newcommand{\nen}{\hspace{-5pt}}
\newcommand{\nquad}{\hspace{-10pt}}
\newcommand{\nqquad}{\hspace{-20pt}}

% Miscellaneous list of math operators I've collected over the years
\DeclareMathOperator{\im}{im}
\DeclareMathOperator{\diag}{diag}
\DeclareMathOperator{\id}{id}
\DeclareMathOperator{\Aut}{Aut}
\DeclareMathOperator{\lcm}{lcm}
\DeclareMathOperator{\Inn}{Inn}
\DeclareMathOperator{\chara}{char}
\DeclareMathOperator{\sgn}{sgn}
\DeclareMathOperator{\rad}{rad}
\DeclareMathOperator{\Hom}{Hom}
\DeclareMathOperator{\End}{End}
\DeclareMathOperator{\Sym}{Sym}
\DeclareMathOperator{\Ant}{Ant}
\DeclareMathOperator{\Int}{int}
\DeclareMathOperator{\supp}{supp}
\DeclareMathOperator{\rank}{rank}
\DeclareMathOperator{\Lie}{Lie}
\DeclareMathOperator{\tr}{tr}
\DeclareMathOperator{\tdet}{det}
\DeclareMathOperator{\Li}{Li}
\DeclareMathOperator{\Br}{Br}
\DeclareMathOperator{\ceil}{ceil}
\DeclareMathOperator{\arccosh}{arccosh}
\DeclareMathOperator{\arctanh}{arctanh}
\DeclareMathOperator{\erf}{erf}
\DeclareMathOperator{\med}{med}

% Environment for the copyright page
\newenvironment{copyrightpage}{\footnotesize\noindent\ignorespaces}{\thispagestyle{empty}}

\newcommand{\institution}{University of Helsinki}
\newcommand{\faculty}{Faculty of Science}
\newcommand{\department}{Department of Physics}

\newcommand{\defenseplace}{\textcolor{\typographersred}{[place]}}
\newcommand{\defensedate}{March 6, 2025}
\newcommand{\defensetime}{\textcolor{\typographersred}{[time]}}

\newcommand{\thesistitle}{Dark Matter in Next Generation Detectors}
\newcommand{\thesisauthor}{Sebastian Sassi}
\newcommand{\thesisyear}{\the\year}

\newcommand{\hipseriesnumber}{\textcolor{\typographersred}{HIP-\thesisyear-XX}}
\newcommand{\isbnp}{\textcolor{\typographersred}{XXX-XXX-XXX-XXX-XXX}}
\newcommand{\isbne}{\textcolor{\typographersred}{XXX-XXX-XXX-XXX-XXX}}
\newcommand{\issnp}{\textcolor{\typographersred}{XXXX-XXXX}}
\newcommand{\issne}{\textcolor{\typographersred}{XXXX-XXXX}}

% End of general stuff everything below is very specific to my thesis.

\newcommand{\bubblestyle}{dashdotted}
\newcommand{\liquidnoblestyle}{solid}
\newcommand{\semiconductorstyle}{dashed}
\newcommand{\scintcrystalstyle}{densely dotted}

\newcommand{\naifillcolor}{scp-red-light-1}

\newcommand{\naicolor}{scp-red-dark-1}
\newcommand{\cawocolor}{scp-purple-dark-1}
\newcommand{\cfcolor}{scp-green-dark-1}
\newcommand{\xenoncolor}{scp-orange-dark-1}
\newcommand{\argoncolor}{scp-brown-dark-1}
\newcommand{\gecolor}{scp-blue-dark-1}
\newcommand{\sicolor}{scp-grey-dark-1}

\newcommand{\wccolor}{scp-orange-dark-1}
\newcommand{\diamondcolor}{scp-green-dark-1}
\newcommand{\siccolor}{scp-purple-dark-1}
\newcommand{\sapphirecolor}{scp-red-dark-1}
\newcommand{\wcolor}{scp-brown-dark-1}

\newcommand{\nufogcolor}{scp-grey-light-2}

\newcommand{\ppcnocolor}{scp-blue-dark-1}
\newcommand{\lineneutrinocolor}{scp-orange-dark-1}
\newcommand{\hepbcolor}{scp-grey-dark-1}
\newcommand{\atmocolor}{scp-purple-dark-1}
\newcommand{\dsnbreactorcolor}{scp-grey-light-2}

\newcommand{\primarylinecolor}{scp-blue-dark-1}
\newcommand{\secondarylinecolor}{scp-orange-dark-1}

\newcommand{\infernoaxiscolor}{scp-grey-light-3}

\newcommand{\darkmarkcolor}{scp-grey-dark-4}

\newcommand{\plotbgcolor}{scp-grey-light-2!50!white}
\newcommand{\plotfgcolor}{white}

% Words LaTeX doesn't know how to hyphenate
\hyphenation{an-i-so-trop-ic}